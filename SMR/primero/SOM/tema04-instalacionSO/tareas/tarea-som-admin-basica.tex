% Created 2025-03-06 Thu 17:30
% Intended LaTeX compiler: pdflatex
\documentclass[11pt]{article}
\usepackage[utf8]{inputenc}
\usepackage[T1]{fontenc}
\usepackage{graphicx}
\usepackage{longtable}
\usepackage{wrapfig}
\usepackage{rotating}
\usepackage[normalem]{ulem}
\usepackage{amsmath}
\usepackage{amssymb}
\usepackage{capt-of}
\usepackage{hyperref}
\usepackage{minted}
\usepackage[margin=0.5in]{geometry}
\hypersetup{colorlinks, linkcolor=black}
\usepackage{geometry}
\geometry{top=2cm}
\geometry{bottom=2cm}
\usepackage{fancyhdr}
\pagestyle{fancy}
\fancyhf[L]{SOM}
\fancyhf[R]{Administración básica de Windows 10}
\fancyfoot[R]{ismaelmacarenochouikh1@gmail.com}
\fancyfoot[L]{CC BY-NC-SA 4.0}
\usepackage{parskip}
\usepackage{mdframed}
\usepackage{fancyvrb}
\usepackage{xcolor}
\definecolor{shadecolor}{RGB}{220,220,220}
\newenvironment{shadedcode}{%
\VerbatimEnvironment
\begin{mdframed}[backgroundcolor=shadecolor,linewidth=0pt]}%
{\end{mdframed}}
\usepackage{attachfile2}
\newcommand{\textattachfilecolor}[2]{\textattachfile[color=0 0 0.5]{#1}{\textcolor{blue}{#2}}}
\usepackage[spanish]{babel}
\usepackage{datetime2}
\DTMlangsetup[es-ES]{ord=raise}
\renewcommand{\dateseparator}{/}
\usepackage{titlesec}
\usepackage{afterpage}
\newcommand\blankpage{\null\thispagestyle{empty}\newpage}
\usepackage{colortbl}
\usepackage{pdfpages}
\usepackage{tcolorbox}
\usepackage{listings}
\usepackage[spanish]{babel}
\lstset{
inputencoding=utf8,
extendedchars=true,
literate={ñ}{{\~n}}1 {Ñ}{{\~N}}1 {á}{{\'a}}1 {é}{{\'e}}1 {í}{{\'i}}1 {ó}{{\'o}}1 {ú}{{\'u}}1 {Á}{{\'A}}1 {É}{{\'E}}1 {Í}{{\'I}}1 {Ó}{{\'O}}1 {Ú}{{\'U}}1,
basicstyle=\ttfamily,
breaklines=true,
columns=fullflexible,
keepspaces=true,
language=TeX,
morekeywords={*, -, **, /}
}
\author{Ismael Macareno Chouikh}
\date{\today}
\title{Práctica sobre Admin. básica de \emph{Windows} 10 (Parte I)}
\hypersetup{
 pdfauthor={Ismael Macareno Chouikh},
 pdftitle={Práctica sobre Admin. básica de \emph{Windows} 10 (Parte I)},
 pdfkeywords={},
 pdfsubject={},
 pdfcreator={Emacs 29.4 (Org mode 9.6.15)}, 
 pdflang={Spanish}}
\begin{document}

\maketitle
\tableofcontents

\blankpage

\section{Caso práctico I - inicio de sistema}
\label{sec:org2467d50}
\begin{enumerate}
\item Accede a la máquina virtual en la que instalaste Windows 10. Captura la forma en la que te pide que inicies sesión
\end{enumerate}

\section{Caso práctico II - configuraciones iniciales}
\label{sec:org28e8999}
\begin{enumerate}
\item Accede al nombre del equipo y cámbialo para que se llame:
\begin{itemize}
\item W10-XX, siendo XX el número de puesto de tu equipo
\end{itemize}
\item Accede al panel de control, Sistema, Administrador de dispositivos, ¿están instalados todos los controladores del equipo? ¿Cómo lo sabes?
\item Selecciona las propiedades de un dispositivo y comprueba su configuración de controlador
\item Indica varias formas de acceder a la configuración de la tarjeta de red del equipo
\item Comprueba la configuración por defecto de la red y de la tarjeta de red del equipo.
\begin{itemize}
\item La red debe ser de Trabajo o Privada, la configuración de tarjeta de red por DHCP, si no es así, cámbialo
\end{itemize}
\item Accede a Windows \emph{Update}
\begin{itemize}
\item Comprueba la configuración
\item ¿Qué opciones hay?
\end{itemize}
\item ¿Se ha instalado algún antivirus por defecto?
\end{enumerate}

\section{Caso práctico III - atajos y características gráficas}
\label{sec:org07acbcd}
\begin{enumerate}
\item Prueba y explica lo que hacen los siguientes atajos:
\begin{itemize}
\item \texttt{Win+E}
\item \texttt{Win+F}
\item \texttt{Win+D}
\item \texttt{Win+L}
\item \texttt{Win+R}
\item \texttt{Win+TAB}
\item \texttt{Win+Flechas}
\end{itemize}
\item ¿Conoces y utilizas alguno más? ¿Cuál?
\end{enumerate}
\end{document}
