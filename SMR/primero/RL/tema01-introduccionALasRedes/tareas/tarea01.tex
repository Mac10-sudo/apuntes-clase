% Created 2025-01-08 Wed 16:26
% Intended LaTeX compiler: pdflatex
\documentclass[11pt]{article}
\usepackage[utf8]{inputenc}
\usepackage[T1]{fontenc}
\usepackage{graphicx}
\usepackage{longtable}
\usepackage{wrapfig}
\usepackage{rotating}
\usepackage[normalem]{ulem}
\usepackage{amsmath}
\usepackage{amssymb}
\usepackage{capt-of}
\usepackage{hyperref}
\usepackage{minted}
\usepackage[margin=0.5in]{geometry}
\hypersetup{colorlinks, linkcolor=black}
\usepackage{geometry}
\geometry{top=2cm}
\geometry{bottom=2cm}
\usepackage{fancyhdr}
\pagestyle{fancy}
\fancyhf[L]{Redes de área local}
\fancyhf[R]{Tema 01}
\fancyfoot[R]{ismael.macareno@educa.madrid.org}
\fancyfoot[L]{CC BY-NC-SA 4.0}
\usepackage{parskip}
\usepackage{mdframed}
\usepackage{fancyvrb}
\usepackage{xcolor}
\definecolor{shadecolor}{RGB}{220,220,220}
\newenvironment{shadedcode}{%
\VerbatimEnvironment
\begin{mdframed}[backgroundcolor=shadecolor,linewidth=0pt]}%
{\end{mdframed}}
\usepackage{attachfile2}
\newcommand{\textattachfilecolor}[2]{\textattachfile[color=0 0 0.5]{#1}{\textcolor{blue}{#2}}}
\usepackage[spanish]{babel}
\usepackage{datetime2}
\DTMlangsetup[es-ES]{ord=raise}
\renewcommand{\dateseparator}{/}
\usepackage{titlesec}
\usepackage{afterpage}
\newcommand\blankpage{\null\thispagestyle{empty}\newpage}
\usepackage{colortbl}
\usepackage{pdfpages}
\usepackage{tcolorbox}
\usepackage{listings}
\usepackage[spanish]{babel}
\lstset{
inputencoding=utf8,
extendedchars=true,
literate={ñ}{{\~n}}1 {Ñ}{{\~N}}1 {á}{{\'a}}1 {é}{{\'e}}1 {í}{{\'i}}1 {ó}{{\'o}}1 {ú}{{\'u}}1 {Á}{{\'A}}1 {É}{{\'E}}1 {Í}{{\'I}}1 {Ó}{{\'O}}1 {Ú}{{\'U}}1,
basicstyle=\ttfamily,
breaklines=true,
columns=fullflexible,
keepspaces=true,
language=TeX,
morekeywords={*, -, **, /}
}
\author{Ismael Macareno Chouikh}
\date{\today}
\title{Práctica 1 - Investigación}
\hypersetup{
 pdfauthor={Ismael Macareno Chouikh},
 pdftitle={Práctica 1 - Investigación},
 pdfkeywords={},
 pdfsubject={},
 pdfcreator={Emacs 29.4 (Org mode 9.6.15)}, 
 pdflang={Spanish}}
\begin{document}

\maketitle
\tableofcontents

\blankpage

\section{Pautas generales con respecto a la entrega de la práctica}
\label{sec:org5efc898}
La práctica deberá contener lo siguiente con respecto a la estética:
\begin{itemize}
\item Encabezados y pies de página
\item Numeración de las páginas
\item Títulos por secciones
\item Portada, índice y bibliografía
\item Texto justificado en tamaño 12
\end{itemize}

\textbf{La práctica se debe entregar en formato PDF obligatoriamente en el aula virtual proporcionada por el docente}  
\section{Generaciones}
\label{sec:org702e6e2}
Investiga más a fondo las distintas generaciones vistas en los \href{https://clasesmaca.gnomio.com/pluginfile.php/29/mod\_resource/content/1/APUNTES\%20RL\%20TEMA\%201.pdf}{apuntes}.

Las generaciones a investigar son:
\begin{itemize}
\item 1ª: lenguaje máquina
\item 2ª: Ensamblador
\item 3ª: Fortran y Cobol
\item 4ª: C, Basic, Pascal, etc.
\item 5ª: Sistemas expertos
\item 6ª: Java, C++, PHP, etc.
\end{itemize}

\section{Resumen de los elementos de la comunicación}
\label{sec:org3ce9178}
Resume de una manera adecuada los elementos necesarios para la comunicación vistos en los apuntes.

Los métodos recomendados por el docente son:
\begin{itemize}
\item Mapa mental
\item Resumen mediante puntos y subsecciones de los mismos
\end{itemize}



\section{Investigación sobre las distintas topologías físicas}
\label{sec:orgd5ed89f}
Investiga sobre las distintas topologías físicas vistas en los apuntes del tema 01.

Investiga sobre:
\begin{itemize}
\item Área de expansión
\item Año de creación
\item Información general
\end{itemize}


\section{Diferencias entre M.A.Cs}
\label{sec:org5760274}
Hacer un cuadro resumen de diferencias entre:
\begin{itemize}
\item M.A.C --> con respecto al identificador único de las tarjetas de red
\item M.A.C --> con respecto a la subcapa MAC de la capa 2 del nivel ISO/OSI (Enlace de datos)
\end{itemize}
\end{document}
