% Created 2025-03-10 Mon 16:16
% Intended LaTeX compiler: pdflatex
\documentclass[11pt]{article}
\usepackage[utf8]{inputenc}
\usepackage[T1]{fontenc}
\usepackage{graphicx}
\usepackage{longtable}
\usepackage{wrapfig}
\usepackage{rotating}
\usepackage[normalem]{ulem}
\usepackage{amsmath}
\usepackage{amssymb}
\usepackage{capt-of}
\usepackage{hyperref}
\usepackage{minted}
\usepackage[margin=0.5in]{geometry}
\hypersetup{colorlinks, linkcolor=black}
\usepackage{geometry}
\geometry{top=2cm}
\geometry{bottom=2cm}
\usepackage{fancyhdr}
\pagestyle{fancy}
\fancyhf[L]{Redes de área local}
\fancyhf[R]{Tema 04 - Cableado estructurado}
\fancyfoot[R]{ismaelmacarenochouikh1@gmail.com}
\fancyfoot[L]{CC BY-NC-SA 4.0}
\usepackage{parskip}
\usepackage{mdframed}
\usepackage{fancyvrb}
\usepackage{xcolor}
\definecolor{shadecolor}{RGB}{220,220,220}
\newenvironment{shadedcode}{%
\VerbatimEnvironment
\begin{mdframed}[backgroundcolor=shadecolor,linewidth=0pt]}%
{\end{mdframed}}
\usepackage{attachfile2}
\newcommand{\textattachfilecolor}[2]{\textattachfile[color=0 0 0.5]{#1}{\textcolor{blue}{#2}}}
\usepackage[spanish]{babel}
\usepackage{datetime2}
\DTMlangsetup[es-ES]{ord=raise}
\renewcommand{\dateseparator}{/}
\usepackage{titlesec}
\usepackage{afterpage}
\newcommand\blankpage{\null\thispagestyle{empty}\newpage}
\usepackage{colortbl}
\usepackage{pdfpages}
\usepackage{tcolorbox}
\usepackage{listings}
\usepackage[spanish]{babel}
\lstset{
inputencoding=utf8,
extendedchars=true,
literate={ñ}{{\~n}}1 {Ñ}{{\~N}}1 {á}{{\'a}}1 {é}{{\'e}}1 {í}{{\'i}}1 {ó}{{\'o}}1 {ú}{{\'u}}1 {Á}{{\'A}}1 {É}{{\'E}}1 {Í}{{\'I}}1 {Ó}{{\'O}}1 {Ú}{{\'U}}1,
basicstyle=\ttfamily,
breaklines=true,
columns=fullflexible,
keepspaces=true,
language=TeX,
morekeywords={*, -, **, /}
}
\author{Ismael Macareno Chouikh}
\date{\today}
\title{Práctica 1 - Presupuesto cableado estructurado de una biblioteca}
\hypersetup{
 pdfauthor={Ismael Macareno Chouikh},
 pdftitle={Práctica 1 - Presupuesto cableado estructurado de una biblioteca},
 pdfkeywords={},
 pdfsubject={},
 pdfcreator={Emacs 29.4 (Org mode 9.6.15)}, 
 pdflang={Spanish}}
\begin{document}

\maketitle
\tableofcontents

\blankpage

\section{Características para el presupuesto}
\label{sec:orgb533495}
Un posible cliente pide un presupuesto para realizar la instalación de una LAN en una biblioteca.
\begin{itemize}
\item El edificio tiene 50m2 de fachada y 60m2
\item El cableado vertical se aloja en el vano junto al ascensor, al lado de la fotocopiadora
\item No se dispone de suelo ni techo técnico
\item Se desea que la fotocopiadora disponga de punto de red. Debe haber puntos de red en los tres despachos de biblioteca
\item La sala de lectura necesito acceso wifi. Debe haber una antena a menos de 20m de cualquier portátil de la sala
\item En la sala de lectura se instalarán 4 puntos de acceso en cada una de las columnas
\end{itemize}


\section{Instrucciones para realizar la práctica}
\label{sec:org3fe9c4f}
Se deberá entregar un documento PDF bien formado con la siguiente información:
\begin{itemize}
\item Plano dibujado indicando mediante una leyenda que significa cada cosa
\item Una explicación bien detallada sobre cada punto del plano
\item Medidas y razonamiento de cantidades en el presupuesto
\item Presupuesto detallado:
\begin{itemize}
\item Objetos adquiridos con sus respectivos enlaces a las páginas web donde se han visualizado
\item Unidades
\item Precio unitario
\item Total
\end{itemize}
\end{itemize}


\section{Instrucciones de entrega de la práctica}
\label{sec:org52ad082}
La práctica se deberá entregar en tiempo y formato.

El formato será:
\begin{itemize}
\item Encabezado y pie de página
\item Numeración de las páginas
\item Portada, índice y bibliografía
\end{itemize}

Se apreciará también que el presupuesto este realizado mediante \emph{Microsoft} excel.
\end{document}
