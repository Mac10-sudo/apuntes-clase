% Created 2025-02-03 Mon 09:43
% Intended LaTeX compiler: pdflatex
\documentclass[11pt]{article}
\usepackage[utf8]{inputenc}
\usepackage[T1]{fontenc}
\usepackage{graphicx}
\usepackage{longtable}
\usepackage{wrapfig}
\usepackage{rotating}
\usepackage[normalem]{ulem}
\usepackage{amsmath}
\usepackage{amssymb}
\usepackage{capt-of}
\usepackage{hyperref}
\usepackage{minted}
\usepackage[margin=0.5in]{geometry}
\hypersetup{colorlinks, linkcolor=black}
\usepackage{geometry}
\geometry{top=2cm}
\geometry{bottom=2cm}
\usepackage{fancyhdr}
\pagestyle{fancy}
\fancyhf[L]{Redes de área local}
\fancyhf[R]{Examen Temas 1-3}
\fancyfoot[R]{ismael.macareno@educa.madrid.org}
\fancyfoot[L]{CC BY-NC-SA 4.0}
\usepackage{parskip}
\usepackage{mdframed}
\usepackage{fancyvrb}
\usepackage{xcolor}
\definecolor{shadecolor}{RGB}{220,220,220}
\newenvironment{shadedcode}{%
\VerbatimEnvironment
\begin{mdframed}[backgroundcolor=shadecolor,linewidth=0pt]}%
{\end{mdframed}}
\usepackage{attachfile2}
\newcommand{\textattachfilecolor}[2]{\textattachfile[color=0 0 0.5]{#1}{\textcolor{blue}{#2}}}
\usepackage[spanish]{babel}
\usepackage{datetime2}
\DTMlangsetup[es-ES]{ord=raise}
\renewcommand{\dateseparator}{/}
\usepackage{titlesec}
\usepackage{afterpage}
\newcommand\blankpage{\null\thispagestyle{empty}\newpage}
\usepackage{colortbl}
\usepackage{pdfpages}
\usepackage{tcolorbox}
\usepackage{listings}
\usepackage[spanish]{babel}
\lstset{
inputencoding=utf8,
extendedchars=true,
literate={ñ}{{\~n}}1 {Ñ}{{\~N}}1 {á}{{\'a}}1 {é}{{\'e}}1 {í}{{\'i}}1 {ó}{{\'o}}1 {ú}{{\'u}}1 {Á}{{\'A}}1 {É}{{\'E}}1 {Í}{{\'I}}1 {Ó}{{\'O}}1 {Ú}{{\'U}}1,
basicstyle=\ttfamily,
breaklines=true,
columns=fullflexible,
keepspaces=true,
language=TeX,
morekeywords={*, -, **, /}
}
\author{Ismael Macareno Chouikh}
\date{\today}
\title{}
\hypersetup{
 pdfauthor={Ismael Macareno Chouikh},
 pdftitle={},
 pdfkeywords={},
 pdfsubject={},
 pdfcreator={Emacs 29.4 (Org mode 9.6.15)}, 
 pdflang={Spanish}}
\begin{document}

\tableofcontents

¿Cuál de las siguientes opciones describe mejor una arquitectura de red?

\begin{itemize}
\item Conjunto de dispositivos de almacenamiento

\item Plan que establece mecanismos para la comunicación entre dispositivos

\item Red de área local restringida

\item Sistema de virtualización para servidores
\end{itemize}

Una arquitectura monolítica en redes:

\begin{itemize}
\item Era compatible con todas las empresas

\item Dividía las funciones en varias capas

\item Abordaba todos los aspectos al mismo tiempo

\item Se enfocaba solo en redes inalámbricas
\end{itemize}

El modelo OSI se compone de:

\begin{itemize}
\item 5 capas

\item 7 capas

\item 4 capas

\item 8 capas
\end{itemize}

La capa física del modelo OSI se encarga de:

\begin{itemize}
\item El direccionamiento IP

\item El transporte de datos entre redes

\item La codificación de información a señales utilizables en el canal

\item La gestión de errores
\end{itemize}

El protocolo TCP es:

\begin{itemize}
\item No orientado a conexión

\item Orientado a conexión

\item Exclusivo para redes inalámbricas

\item Usado solo para broadcast
\end{itemize}

La función principal de la capa de enlace es:

\begin{itemize}
\item Definir el mejor camino para la transmisión de datos

\item Garantizar una comunicación libre de errores entre dispositivos directamente conectados

\item Proporcionar servicios de cifrado

\item Gestionar el direccionamiento IP
\end{itemize}

¿Cuál es la unidad de datos (PDU) de la capa de red?

\begin{itemize}
\item Trama

\item Paquete

\item Segmento

\item Bits
\end{itemize}

El protocolo IP se encarga de:

\begin{itemize}
\item Gestión de sesiones Cifrado de datos

\item Encaminamiento de paquetes

\item Sincronización de dispositivos
\end{itemize}

Una red PAN está diseñada para:

\begin{itemize}
\item Conectar dispositivos a nivel personal

\item Conectar dispositivos a través de ciudades

\item Interconectar continentes Transmitir datos entre varios edificios
\end{itemize}

En una topología en estrella:

\begin{itemize}
\item Todos los nodos están conectados en un lazo cerrado

\item Si falla el nodo central, la red cae

\item Los datos circulan sin nodo central

\item No hay colisiones posibles
\end{itemize}

La transmisión en modo simplex se caracteriza por:
\begin{itemize}
\item Comunicación simultánea en ambos sentidos

\item Comunicación en ambos sentidos pero alternada

\item Comunicación en un solo sentido

\item Transmisión mediante señales cifradas
\end{itemize}

En una red punto a punto:

\begin{itemize}
\item Todos los nodos comparten el mismo canal

\item Se establece una conexión dedicada entre dos nodos

\item La información se transmite mediante token

\item Los paquetes no requieren direcciones
\end{itemize}

El encaminamiento consiste en:

\begin{itemize}
\item Determinar la mejor ruta para transmitir datos

\item Dividir los datos en tramas

\item Gestión de errores en la capa de enlace

\item Configurar el hardware de red
\end{itemize}

La multiplexación permite:

\begin{itemize}
\item Reducir errores de transmisión

\item Dividir una red en varias subredes

\item Compartir un canal para múltiples comunicaciones

\item Conectar dispositivos inalámbricos
\end{itemize}

La función de un switch en una red es:

\begin{itemize}
\item Repetir señales a todos los puertos

\item Encaminamiento entre redes

\item Conectar nodos y redirigir tramas al destino correcto

\item Manejar conexiones VPN
\end{itemize}

El protocolo DNS se utiliza para:

\begin{itemize}
\item Transferencia de archivos

\item Resolución de nombres de dominio

\item Cifrado de datos

\item Gestión de direcciones IP
\end{itemize}

La ventaja principal de una red privada es:

\begin{itemize}
\item Bajo coste

\item Alta disponibilidad pública

\item Diseño a medida y control total

\item Bajo mantenimiento
\end{itemize}

Una red MAN suele abarcar:

\begin{itemize}
\item Un edificio

\item Una región o conjunto de edificios

\item Varias ciudades

\item Dispositivos personales
\end{itemize}

El protocolo UDP:

\begin{itemize}
\item Garantiza la entrega de datos

\item Es orientado a conexión

\item No garantiza la entrega de datos

\item Siempre cifra la información
\end{itemize}

La capa de presentación del modelo OSI se encarga de:

\begin{itemize}
\item Encaminamiento de paquetes

\item Compresión y cifrado de información

\item Establecer la conexión física

\item Gestionar el acceso al medio
\end{itemize}

Un ejemplo de protocolo en la capa de aplicación es:

\begin{itemize}
\item IP

\item FTP

\item TCP

\item Ethernet
\end{itemize}

El modelo TCP/IP tiene:

\begin{itemize}
\item Tres capas

\item Cuatro capas

\item Cinco capas

\item Siete capas
\end{itemize}

La capa de transporte en TCP/IP se encarga de:

\begin{itemize}
\item Direccionamiento lógico

\item Encaminamiento

\item Gestión de la conexión extremo a extremo

\item Gestión de direcciones físicas
\end{itemize}

En una topología de bus:

\begin{itemize}
\item Los nodos se conectan mediante un nodo central

\item Los nodos se conectan en serie formando un anillo

\item Los nodos comparten un único cable

\item La información siempre viaja en paralelo
\end{itemize}

La función principal de un router es:

\begin{itemize}
\item Repetir señales

\item Encaminamiento de paquetes entre redes

\item Almacenar datos temporalmente

\item Gestionar el acceso al medio
\end{itemize}

En una conexión half-duplex:

\begin{itemize}
\item La comunicación es en un solo sentido

\item La comunicación es simultánea en ambos sentidos

\item La comunicación es alternada entre ambos nodos

\item Solo se transmiten datos cifrados
\end{itemize}

El control de flujo se refiere a:

\begin{itemize}
\item La gestión del direccionamiento IP

\item La regulación de la velocidad de transmisión entre emisor y receptor

\item La división de paquetes grandes

\item La selección del mejor camino
\end{itemize}

El sistema DSL permite:

\begin{itemize}
\item Comunicaciones por radiofrecuencia

\item Alta velocidad en cableado de baja calidad

\item Transmisión de datos solo en redes privadas

\item Uso exclusivo para voz
\end{itemize}

Una VPN ofrece:

\begin{itemize}
\item Comunicaciones públicas sin seguridad

\item Ilusión de una LAN única mediante redes públicas

\item Solo conexiones locales

\item Comunicación inalámbrica
\end{itemize}

La multiplexación por división de tiempo:

\begin{itemize}
\item Divide el canal en bandas de frecuencia

\item Divide el tiempo en ranuras asignadas a cada usuario

\item No permite transmisiones simultáneas

\item Solo se usa en redes inalámbricas
\end{itemize}

La PDU de la capa física es:

\begin{itemize}
\item Bits

\item Trama

\item Paquete

\item Segmento
\end{itemize}

Una red RDSI permite:

\begin{itemize}
\item Solo transmisión de voz

\item Voz y datos multiplexados

\item Solo conexiones inalámbricas

\item Comunicaciones exclusivas para voz analógica
\end{itemize}

La codificación Manchester diferencia un bit 0 y 1 mediante:

\begin{itemize}
\item Cambio de fase al principio del bit

\item Transición a mitad del bit

\item Ausencia de transiciones

\item Mantenimiento de la polaridad
\end{itemize}

La multiplexación por división de frecuencia asigna:

\begin{itemize}
\item Ranuras de tiempo

\item Bandas de frecuencia

\item Direcciones IP

\item Paquetes segmentados
\end{itemize}

Un bridge conecta:

\begin{itemize}
\item Dos segmentos de red

\item Dispositivos inalámbricos

\item Servidores web

\item Redes internacionales
\end{itemize}

El ruido en una transmisión es:

\begin{itemize}
\item Una deformación predecible

\item Interferencia aleatoria y no deseada

\item Pérdida de paquetes

\item Cambio de fase controlado
\end{itemize}

El protocolo FTP se utiliza para:

\begin{itemize}
\item Transferencia de archivos

\item Resolución de nombres

\item Gestión de sesiones

\item Control de accesos
\end{itemize}

La transmisión asíncrona requiere:

\begin{itemize}
\item Una velocidad constante

\item Señales de sincronización incluidas en el mensaje

\item Relojes sincronizados

\item Cifrado extremo a extremo
\end{itemize}

Una topología en árbol es una extensión de:

\begin{itemize}
\item Bus

\item Estrella

\item Anillo

\item Jerárquica
\end{itemize}

Un concentrador (hub) en una red:

\begin{itemize}
\item Redirige paquetes al destino correcto

\item Repite señales a todos los puertos

\item Filtra tramas por dirección

\item Encapsula datos
\end{itemize}

La dirección IP pertenece a la capa:

\begin{itemize}
\item Física

\item Enlace

\item Red

\item Transporte
\end{itemize}

La transmisión de voz por corriente eléctrica se realiza en:

\begin{itemize}
\item RTC

\item VPN

\item DSL

\item ISDN
\end{itemize}

La transformada de Fourier permite:

\begin{itemize}
\item Reducir el ruido de transmisión

\item Identificar las frecuencias componentes de una señal

\item Mejorar el ancho de banda

\item Optimizar las conexiones
\end{itemize}

El control de acceso al medio (MA- se encuentra en:

\begin{itemize}
\item Capa física

\item Subcapa de enlace

\item Capa de transporte

\item Capa de aplicación
\end{itemize}

La atenuación en una transmisión provoca:

\begin{itemize}
\item Deformación de la señal

\item Pérdida de intensidad de la señal

\item Interferencia con otras señales

\item Mejora en la calidad
\end{itemize}

La transmisión paralela:

\begin{itemize}
\item Envía un bit por vez

\item Envía varios bits simultáneamente

\item Requiere modems especiales

\item Solo se usa en redes inalámbricas
\end{itemize}

El protocolo Telnet permite:

\begin{itemize}
\item Resolución de nombres

\item Transferencia de archivos

\item Conexión remota

\item Control de acceso
\end{itemize}

Una red WAN abarca:

\begin{itemize}
\item Un área local

\item Una región pequeña

\item Varias ciudades o países

\item Solo dispositivos personales
\end{itemize}

El direccionamiento MAC es:

\begin{itemize}
\item Dirección lógica

\item Dirección física

\item Dirección virtual

\item Dirección de transporte
\end{itemize}

La capa LLC pertenece a:

\begin{itemize}
\item Capa física

\item Subcapa de enlace

\item Capa de red

\item Capa de transporte
\end{itemize}
\end{document}
