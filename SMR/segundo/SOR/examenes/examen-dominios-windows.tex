% Created 2025-02-11 Tue 18:06
% Intended LaTeX compiler: pdflatex
\documentclass[11pt]{article}
\usepackage[utf8]{inputenc}
\usepackage[T1]{fontenc}
\usepackage{graphicx}
\usepackage{longtable}
\usepackage{wrapfig}
\usepackage{rotating}
\usepackage[normalem]{ulem}
\usepackage{amsmath}
\usepackage{amssymb}
\usepackage{capt-of}
\usepackage{hyperref}
\usepackage{minted}
\usepackage[margin=0.5in]{geometry}
\hypersetup{colorlinks, linkcolor=black}
\usepackage{geometry}
\geometry{top=2cm}
\geometry{bottom=2cm}
\usepackage{fancyhdr}
\pagestyle{fancy}
\fancyhf[L]{Sistemas Operativos en Red}
\fancyhf[R]{Examen controlador de dominio de Windows}
\fancyfoot[R]{ismael.macareno@educa.madrid.org}
\fancyfoot[L]{CC BY-NC-SA 4.0}
\usepackage{parskip}
\usepackage{mdframed}
\usepackage{fancyvrb}
\usepackage{xcolor}
\definecolor{shadecolor}{RGB}{220,220,220}
\newenvironment{shadedcode}{%
\VerbatimEnvironment
\begin{mdframed}[backgroundcolor=shadecolor,linewidth=0pt]}%
{\end{mdframed}}
\usepackage{attachfile2}
\newcommand{\textattachfilecolor}[2]{\textattachfile[color=0 0 0.5]{#1}{\textcolor{blue}{#2}}}
\usepackage[spanish]{babel}
\usepackage{datetime2}
\DTMlangsetup[es-ES]{ord=raise}
\renewcommand{\dateseparator}{/}
\usepackage{titlesec}
\usepackage{afterpage}
\newcommand\blankpage{\null\thispagestyle{empty}\newpage}
\usepackage{colortbl}
\usepackage{pdfpages}
\usepackage{tcolorbox}
\usepackage{listings}
\usepackage[spanish]{babel}
\lstset{
inputencoding=utf8,
extendedchars=true,
literate={ñ}{{\~n}}1 {Ñ}{{\~N}}1 {á}{{\'a}}1 {é}{{\'e}}1 {í}{{\'i}}1 {ó}{{\'o}}1 {ú}{{\'u}}1 {Á}{{\'A}}1 {É}{{\'E}}1 {Í}{{\'I}}1 {Ó}{{\'O}}1 {Ú}{{\'U}}1,
basicstyle=\ttfamily,
breaklines=true,
columns=fullflexible,
keepspaces=true,
language=TeX,
morekeywords={*, -, **, /}
}
\author{Ismael Macareno Chouikh}
\date{\today}
\title{Examen Controlador de dominio de Windows}
\hypersetup{
 pdfauthor={Ismael Macareno Chouikh},
 pdftitle={Examen Controlador de dominio de Windows},
 pdfkeywords={},
 pdfsubject={},
 pdfcreator={Emacs 29.4 (Org mode 9.6.15)}, 
 pdflang={Spanish}}
\begin{document}

\maketitle

\section{Ejercicio 1}
\label{sec:org68734ab}
\textbf{Previo:} configurar en tarjeta de red, el DNS la propia IP de Windows 2019.

Promocionar a controlador de dominio Windows Server 2019:
\begin{itemize}
\item En vuestro caso, va a ser el único dominio que vamos a crear en vuestra organización, le llamáis \textbf{Empresa\textsubscript{INICIALESVUESTRAS.es}}
\item Crear con nivel funcional 2016
\item Configurar como contraseña de restauración "Rescate123"
\item Una vez instalado, reiniciamos
\end{itemize}


\section{Ejercicio 2}
\label{sec:org2b82291}
\begin{enumerate}
\item Añadir dos máquinas, cliente1 y cliente2 al dominio
\item Inicia sesión en cliente1 con el usuario administrador del dominio
\item Inicia sesión en cliente2 con el usuario local empleado. Crea una carpeta en el Escritorio. Cierra la sesión
\item Vuelve a iniciar sesión en cliente2 con el usuario global empleado. Crea una carpeta en el Escritorio. Cierra la sesión
\item Ahora, desde el cliente2 con el usuario local supervisor, ve a la carpeta de usuarios y visualiza las distintas carpetas de usuarios existentes. ¿Cómo se diferencian las carpetas
de los usuarios locales con los globales?
\end{enumerate}


\section{Ejercicio 3}
\label{sec:orgdc92c6a}
\begin{enumerate}
\item Crea dos cuentas globales, pedro y ana.
\item Pulsa sobre propiedades del usuario, y observa las diferentes características que se pueden indicar en un usuario, en especial los atributos que se especifican en la solapa \textbf{Cuenta}.
\begin{itemize}
\item Pon una limitación horaria, de forma que no se pueda acceder en la hora actual
\item Deshabilita la cuenta
\item Pon una limitación de acceso, de manera que solo se pueda acceder desde la máquina cliente1
\end{itemize}
\end{enumerate}


\section{Ejercicio 4}
\label{sec:org0368d18}
\begin{enumerate}
\item Comprobar que en búsqueda directa aparecen vuestros equipos
\begin{itemize}
\item ¿Funciona \texttt{nslookup} tanto por nombre como por dirección IP?
\end{itemize}
\item Crea la zona de búsqueda inversa
\item Añadir los punteros \texttt{PTR} de los equipos
\item Marcar actualizar punteros \texttt{PTR}
\item Repetir \texttt{nslookup} por dirección IP. ¿Funciona?
\end{enumerate}


\section{Ejercicio 5}
\label{sec:orgc37d98c}
\begin{enumerate}
\item Crear dos \texttt{OU}\footnote{\emph{Organizational unit}}, una madrid y otra sevilla.
\item Crear dentro de la \texttt{OU} madrid otra \texttt{OU} llamada Electricista
\item Mover el equipo cliente1 a la \texttt{OU} electricista
\item Mover el equipo cliente2 a la \texttt{OU} sevilla
\item Crear un usuario dentro de la \texttt{OU} Electricista llamado electrico
\item Crear un usuario dentro de la \texttt{OU} sevilla llamado sevillano
\item Crear una \texttt{OU} llamada equipos reales y meter en esta a los equipos cliente1 y cliente2
\end{enumerate}


\section{Ejercicio 6}
\label{sec:orgcbfc45a}
\begin{enumerate}
\item Publicar los recursos compartidos en cliente1 (\texttt{rw}) dentro de la \texttt{OU} equiposReales
\item Crear en el controlador de dominio la carpeta carpeta1 en \texttt{C:\textbackslash{}}
\item Compartir carpeta1 con permisos de lectura a usuarios del dominio
\item Publicar carpeta1 en la \texttt{OU} equiposReales
\item Iniciar sesión en cliente2 en el dominio como empleado y busca los recursos compartidos por el dominio
\item Buscar desde cliente2 los equipos del dominio
\item Comprueba que entras a carpeta1 con los permisos configurados
\item Comprobar que entras a los recursos lectura y escritura
\item Reiniciar sesión en cliente2 como usuario local empleado
\item ¿Puedes buscar los recursos compartidos del dominio? ¿Por qué?
\item ¿Puede entrar al recurso lectura de cliente1?
\end{enumerate}


\section{Ejercicio 7}
\label{sec:orgfc9a264}
\textbf{Previo al ejercicio, mover los usuarios pedro y empleado a la \texttt{OU} equiposReales}
\begin{enumerate}
\item Establecer un fondo de escritorio obligatorio para la \texttt{OU} equiposreales
\begin{itemize}
\item Inicia sesión en los distintos clientes y comprueba que se ha establecido el fondo escritorio
\end{itemize}
\end{enumerate}


\section{Ejercicio 8}
\label{sec:org7a3f726}
\begin{enumerate}
\item Establecer política para instalar Firefox a los miembros de una \texttt{OU} en configuración de usuario
\begin{itemize}
\item Comprueba que se puede instalar el \emph{software}
\end{itemize}
\end{enumerate}


\section{Ejercicio 9}
\label{sec:orge55a34f}
\begin{enumerate}
\item Redireccionar documentos, el objetivo, es que independientemente de donde inicie sesión el usuario, la carpeta Documentos de todos los usuarios de la \texttt{OU} se encuentre en el
servidor en \texttt{C:\textbackslash{}DocumentosUsuarios}
\end{enumerate}


\section{Ejercicio 10}
\label{sec:orgfc94f6a}
\begin{enumerate}
\item Impedir el acceso de usuarios distintos a los que están en equiposReales que inicien sesión en cliente1 y cliente2
\end{enumerate}


\section{Ejercicio 11}
\label{sec:org411654c}
\begin{enumerate}
\item A partir de que creamos el dominio, los usuarios normales no pueden entrar en equipo controlador de dominio. Esto se realiza por seguridad del propio controlador de dominio.
Este comportamiento está definido por la política por defecto del dominio.
\begin{itemize}
\item \url{https://blog.soporteti.net/como-permitir-el-inicio-de-sesion-local-en-windows-server-2008-como-controlador-de-dominio/}
\end{itemize}
\end{enumerate}
\end{document}
